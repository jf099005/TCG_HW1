\documentclass[a4paper, 12pt]{article}  %set doc type and font size
\usepackage{fontspec}                   %allow setting fonts
\usepackage{xeCJK}         
\usepackage{amsmath}
\newtheorem{theorem}{Theorem}
\setCJKmainfont{Noto Sans CJK SC}             %call chinese package xeCJK
% \setmainfont{Times New Roman}           %set English font

% \XeTeXlinebreaklocale "zh"
% \XeTeXlinebreakskip = 0pt plus 1pt


\usepackage{algorithm}
\usepackage[noend]{algpseudocode}

\title{電腦對局理論 Homework 1}
\author{蔡平樂}
\date{\today}


\begin{document}
\maketitle

\section{Algorithm design}


主體搜尋使用$IDA^*$配合$DFS1_{cost}$,其中$IDA^*$實作完全與課堂中相同,因此不另外說明。
$DFS$演算法在一些細節上有修改,具體演算法如下:
\begin{algorithm}[!h]
\begin{algorithmic}
    \Procedure{DFS}{$pos$, $threshold$}
        \State$Stack\_init(S)$
        \State$Stack\_init(Prv\_States)$
        \State$Push(S, pos)$
        \State$depth \gets 0$
        \While{\textbf{not} $S.empty()$}
            \State$current\gets S.pop()$
            \If{$current == Null$}
                \State$depth--$
                \State\textbf{continue}
            \EndIf
            \State$record(current, limit\_depth - depth)$
            \State$NX \gets next\_states\_gen(current.pos)$
            \State$Next \gets sort\_and\_erase(NX, maximum = threshold - depth - 1)$
            
            \State$Push(S, Null)$
            \For{$nx$ in Next}
                \If{$is\_visited(nx)$} \textbf{continue}
                \EndIf
                \If{$is\_win(nx)$} \textbf{return true}
                \EndIf
                \State$Push(S, nx)$
            \EndFor
            \State$depth++$

        \EndWhile
        \State\textbf{return false}
    \EndProcedure
\end{algorithmic}
\end{algorithm}

細節將於其後說明

\subsection{$sort\_and\_erase$}
$sort\_and\_erase(NX, maximum)$使用Heuristic value排序$NX$,\\
並移除所有$value > maximum$的元素,因此無須在push階段進行剪枝。


實作上,最初是使用標準算法的使用std::sort後於push過程中剪枝,
但使用gprof時發現heuristic時間佔比相當高,也發現std::sort基本上只調用insertion sort進行排序,
考量到實作insertion sort時間成本較低,且可以大幅降低heuristic function的呼叫次數,
因此自行實作sort函數排序$NX$,使用略帶修改的insertion sort,
使其直接忽略$value > threshold$的元素並能在同時計算$\le threshold$的元素數量,
實測在公佈的3-3測資上能夠將約60秒的時間壓縮至15秒內,常數上有相當大的優化。

\subsection{$record$ / $is\_visited$}
在搜索完每個$pos$後,將$(pos, remain\_depth)$紀錄,代表$pos$在深度$remain\_depth$下不可能解開。
$is\_visit(pos, remain\_depth)$則會判斷是否盤面$pos$是否已搜索過深度$\le remain\_depth$的狀態,若是則return true。
具體實作如下:

\begin{algorithm}
    \begin{algorithmic}
        \Procedure{record}{pos, remain\_depth}
            \If{$pos \in recorder.keys$}
                \State$recorder[pos] \gets \max(recorder[pos], remain\_depth)$
            \Else
                \State $recorder[pos] \gets remain\_depth$
            \EndIf
        \EndProcedure

        \Procedure{$is\_visited$}{$pos, remain\_depth$}
            \If{$pos \notin recorder.keys$}
                \textbf{return false}
            \Else
                \textbf{ return} $recorder[pos] \ge remain\_depth$
            \EndIf
        \EndProcedure
    \end{algorithmic}
\end{algorithm}



\newpage
\section{Heuristic function design}

\subsection{algorithm}

\begin{algorithmic}[] % [1] 表示每行編號
    \Procedure{MinimumMoveEstimate}{pos}
    \State$Priority\_Queue\_init(PQ)$

    \For{$b$ in $Black\_Pieces(pos)$}
        \For{$r$ in $Red\_Pieces(pos)$}
            \If {$b$ can capture $r$}
                \State$EnPriority\_Queue(PQ, dis(b,r))$
            \EndIf
        \EndFor
    \EndFor


    \For{$(r_1, r_2)$ be any pair in $Red\_Pieces(pos)$}
        \For{$r_2$ in $Red\_Pieces(pos)$}
            \State$EnPriority\_Queue(PQ, dis(r_1,r_2))$
        \EndFor
    \EndFor

    \State$N = |Red\_Pieces(pos)|$

    \State$p = sum\{{\rm first\ N\ elements\ in\ PQ}\}$

    \State\textbf{return} $x$
    \EndProcedure
\end{algorithmic}

$dis(sq\_a, sq\_b)$ 為位置 $sq\_a$ 到位置 $sq\_b$的最短路徑,在沒有Duck的情形下為Euclidean distance,在有Duck時使用Floyd-Warshall algorithm計算。



\subsection{admissible proof}
Fix a board state $pos$, suppose $OPT$ is the exact minimum step to capture all red pieces.\\
First, we transform a Board $pos$ into a weighted, edge-colored graph $G$ by following:

Let $B = \{b_1,b_2,...,b_n\}$ be the set of all black pieces, 
and $R = \{r_1,r_2,...,r_m\}$ be the  set of all red pieces.
Then $G=(V,E)$ with $E=B\cup R$, node of $B$ and

\begin{itemize}
    \item if $b_i\in B$ can capture $r_j\in R$, then $(b_i, r_j)\in E$ with color Black and weight $dis(b_i.location, r_j.location)$
    \item foreach $r_i,r_j\in R$, then $(r_i, r_j)\in E$\\
            with color Red and weight $dis(r_i.location, r_j.location)$
\end{itemize}


Suppose we have an optimal solution.\\
Foreach single $b_k$, define $M_k$ as the move of $b_k$ in the optimal solution, and $R_k = \{r^k_1, r^k_2,...,r^k_{n_k} \}$ be the red pieces captured by $b_k$ sequentially in the optimal solution, 
then by the proposition of optimal solution, we have following proposition
\begin{itemize}
    \item $(b_k,r^k_1)\in E, (r^k_i, r_{i+1}^k)\in E\forall i<n_k+1$
    \item $\bigcup_k R_k = R$, each $R_k$ disjoint.
    \item Let $P_k=\{(b_k, r^k_1), (r^k_1, r^k_2),...,(r^k_{n_k-1}, r^k_{n_k})\}$ be the path, then the union of all paths will cover all red nodes.
    i.e. $R\subseteq \cup_{k=1}^n P_k$
\end{itemize}

% Then we will prove $OPT$ less or equal the solution of following problem
% \begin{align*}
%     {\rm Find\ paths} P_1,P_2,...,P_M {\rm with}  
% \end{align*}



from triangular equation we have

\begin{align*}
    M_k \ge weight(b_k, r^k_1) + \sum_{j=1}^{n_k-1} weight(r^k_j,r^k_{j+1})
\end{align*}

Therefore

\begin{equation*}
    Optimal\ solution\ steps = \sum_{k=1}^n M_k \ge \sum_{k=1}^n weight(b_k, r^k_1) + \sum_{j=1}^{n_k-1} weight(r^k_j,r^k_{j+1})
\end{equation*}

$\sum_{k=1}^n weight(b_k, r^k_1) + \sum_{j=1}^{n_k-1} weight(r^k_j,r^k_{j+1})$ 
is the sum of $|R|$ edges in $G$, and it will smaller or equal than the sum of $|R|$ edges
with minimum weight at $G$.

\subsubsection{Variant of the algorithm}

\end{document}